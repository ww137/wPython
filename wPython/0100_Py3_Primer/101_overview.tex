\section{Python 둘러보기}

이 장에서는 programming language인 파이썬(Python)에 대해서 간단히 둘러본다. 이 책은 Python 3.7을 기준으로 작성되었으며 개발진이 하위 호완성을 포기하지 않을 것임이 명백하므로 그 이후의 판(version)에도 적용될 것이다. 더 자세한 내용은 비록 가독성이 나쁘기로 유명하지만 \url{https://www.python.org/}을 참조하라.
아직도 Python 2.7을 사용하는 곳이 많지만 차츰 Python 3으로 대체되는 중이다. 조만간 지원을 완전히 끊을 것이므로 Python 3으로 시작하는 것이 당연하다고 생각한다.

\subsection{Python Interpreter}

Python은 \emph{interpreted} language이다. 명령(command)은 \emph{Python interpreter}라는 software를 통해 평가된다. 여기서 평가는 것은 evaluate의 번역어이다. Interpreter는 명령을 입력받아서 평가하고 결과를 출력하는 역할을 한다. Interpreter를 해석기라고 번역하는 경우가 있는데 필자는 compiler와 혼동할 수 있으므로 별로 좋아하지 않는다.

C와 같은 언어를 다룬 경험이 있는 독자들은 compiler를 고급 언어를 기계어로 번역하여 computer가 실행할 수 있도록 하는 software라고 알고 있을 것이다. (물론 엄밀히 말하자면 번역기에 가깝지만 이 책의 목적상 생략한다.) Interpreter도 마찬가지 아닌가 생각할 수 있지다. 하지만 interpreter는 명령어를 한줄씩 입력받아서 기계어로 번역하여 실행하는 반면 compiler는 하나의 완성된 명령어를 입력받아서 기계어로 번역한다는 큰 차이점이 존재한다.

Interpreter가 명령을 한줄씩 입력받아서 실행하지만 보통 programmer들은 열련의 명령어를 text file에 저장하여 실행한다. 이러한 text file을 \emph{source code} 또는 \emph{script}라고 한다. 전자는 두루 쓰이는 용어이지만 후자는 interpreted language에만 쓰인다.
보통 Python script는 \verb|.py|라는 확장자를 사용한다.